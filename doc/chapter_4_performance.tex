%! 성능 측정 방법에서는 지금 "대충 이런걸 적을 거에요"를 보여준 거고.
%! 성능 측정 결과에서는 지금 "성능 측정 결과, 이렇게 나왔습니다"를 보여준거다.
%! 이런 내용이 들어갈거에요 정도로 생각해야함.

\chapter{워터마크 성능}

(대충 여기가 뭐하는 장인지 적고, 이 장의 요약 적기)

\section{성능 측정 방법}

(넣은 이유: 성능 측정 결과에 신뢰성을 주기 위함.)

영상의 경우 성능을 측정하기 어려워, 영상 대신 이미지를 수집하여 워터마크를
삽입함. 워터마크가 삽입된 이미지로부터 워터마크가 얼만큼 추출되는지를 확인.

\begin{itemize}
    %? 더 필요한 내용이 있는가? 혹은 필요 없는 내용이나, 고쳐야 할 내용이 있는가?
    %? 구성방식의 문제?
    \item 이미지 수집
    \begin{itemize}
        \item \textbf{몇 개 수집했는가?} - 현재는 600개 (더 수집할 수 있음. 시간 오래 안걸림.)
        %? 1000개 딱 맞출까?
        \item \textbf{어떤 이미지를 수집했는가?} - PPT 사진, 문서 사진, 사람 얼굴이
        나오는 사진을 사용. 화상회의에서는 이러한 화면이 자주 나옴.
        %? 또 해봤으면 하는 사진이 있는가? 예를 들어, 자연 환경이라던가.
        %? 자연환경사진은 화상회의와 관련없지만 정확한 성능 측정에 도움이 될 수는 있다고 생각.
        %! 어떻게 수집했는가?에 대한 내용은 안넣었음.
        %! 사진 깃허브에 올리기: 굳이 안보여줘도 충분히 설득력 있는 글이 될 것이라 생각. 예시로
        %! 사진 몇 장 직접 만들어서 보여주는 것이 좋아보임.
    \end{itemize}
    \item 이미지 전처리
    \begin{itemize}
        \item \textbf{적절한 크기 이미지 사용.} 너무 작은(720p 미만) 이미지나 너무 큰(4k
        초과) 이미지는 제거.
        %! 여기를 <이미지 수집> 단계의 <어떤 이미지를 수집했는가> 단계에 넣을 수도 있음.
        %! 전처리할 내용이 적으면 그렇게 할 예정.
        %! 이미지를 AI를 활용해서 관련없는 이미지를 제거하거나 분류하는 건...시간 상 안될지도...
    \end{itemize}
    \item 워터마크 삽입
    \begin{itemize}
        \item \textbf{메시지와 키는 고정.}  성능 측정을 편하게 하기 위해 메시지와
        키를 -로 고정했음. 따라서 텍스트 워터마크와 qr코드 워터마크 형태는 변하지
        않으며, 모든 사진에 동일한 워터마크가 들어감.
        \item \textbf{애니메이션 효과 제거.} 텍스트 워터마크는 애니메이션 효과가
        있는데, 이는 제거함.
        \item \textbf{그 외에는 전부 동일.} 이전 장에서 설명한 방식과 동일하게
        워터마크 삽입. 투명도는 사용자 설정으로 조절 가능하니, 투명도를 어느정도
        주었는지 설명. 예시 보여주기.
        %! 깃허브에 워터마크 삽입 이미지 올릴수 있음.
        %? 추가 내용이 있을까?
        %! 어떻게 삽입했는가는 필요 없을 듯. 굳이 파이썬의 어떤 모듈을 이용해서 어쩌구 저쩌구. 그럴필요 있나?
        %! 프로토타입이니까 설명해야 할 것 같기도 하고.
    \end{itemize}
    \item 워터마크 훼손
    \begin{itemize}
        \item \textbf{노이즈 삽입.} AI 가 워터마크를 훼손했을 때 성능을 측정하기
        위함. 이미지 위에 작은 원을 골고루 뿌려 QR코드 워터마크를 훼손. 원
        크기를 키워서 크기에 따른 성능을 측정할 예정.
        \item \textbf{이미지 자르기.} 화상회의 화면은 겉을 조금 자르더라도
        가치있는(?) 화면이 될 수 있으므로, 공격자는 영상의 겉을 자를 수 있음.
        그리고 자른 만큼 워터마크가 사라지므로, 이에 대한 성능 측정이 필요.
        따라서 사진의 겉을 잘라서 성능을 측정. 점점 많이 잘랐음.
        \item \textbf{크기 조정.} (현재 이에 대해 성능 측정을 못해서 성능 측정이
        가능하면 적을 예정. 그러나 화상회의 영상은 시간이 지나면서 화질(?)이
        달라질 수 있으므로 이에 대한 워터마크 성능 측정을 하긴 해야함.)
    \end{itemize}
    \item 워터마크 추출 (현재 텍스트 워터마크 추출은 못한 상황...)
    \begin{itemize}
        \item \textbf{QR 워터마크를 어떻게 수집했는가?} QR코드 크기는 일정하기
        때문에 그리드 형태로 쉽게 자를 수 있음. 잘라서 QR코드가 위치한 부분만 모았음.
        \item \textbf{QR 코드 전처리} 자른 QR 코드 사진을 흑백으로 조정. 여기서
        바로 추출을 시도. 추출이 안된 QR코드는 Linear Filter를 이용하여 전처리
        후 다시 추출 시도. 예시 사진 보여주기. Linear Filter가 무엇인지 설명.
        \item \textbf{QR에서 메시지를 어떻게 뽑았는가?} 파이썬의 XX 모듈을 사용.
        (리더기 성능에 따라 추출이 될수도 있고 안될수도 있고 해서 이를
        넣어주었음.)
    \end{itemize}
\end{itemize}

\section{성능 측정 결과}

\iffalse
  어떤 내용을 넣을까? 추상적으로.
  - 추출이 잘 되는가?
    - 원본에서 추출이 잘 되는가?
    - 공격 후 추출이 잘 되는가?

  어떤 내용을 넣을까? 자세하게.
  - 

  영상이 손상되는 경우에는 뭐가 있을까?
\fi